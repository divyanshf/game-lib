\documentclass[12pt]{article} 
%\topmargin=-1in    
\textheight=22cm  
\oddsidemargin=1pt % leftmargin is 1 inch
\textwidth=6.5in   % textwidth of 6.5in leaves 1 inch for right margin
\usepackage{graphicx}
\usepackage{epstopdf}
\usepackage[tight,footnotesize]{subfigure}
\usepackage{eso-pic}
\usepackage{float}
\usepackage{blindtext}
\usepackage[T1]{fontenc}
\usepackage[utf8]{inputenc}
\usepackage{ucs}
\usepackage{amsmath}
\usepackage{amsfonts}
\usepackage{amssymb}
\usepackage{graphicx}
\usepackage{multicol}
\usepackage{hyperref}
\usepackage{url}
\usepackage{color}
\usepackage{lettrine}
\usepackage{enumerate}
\usepackage{newlfont}
%\usepackage{subfig}
%\usepackage[round]{natbib}

\renewcommand{\labelitemi}{$\bullet$}
\renewcommand{\labelitemii}{$\cdot$}
\renewcommand{\labelitemiii}{$\diamond$}
\renewcommand{\labelitemiv}{$\ast$}
\renewcommand{\thefootnote}{\fnsymbol{footnote}}
% for circle $\circ$
% set margin -------------------------------------------------------------
\oddsidemargin=-.1in
\evensidemargin=.1in
\textwidth=6.2in
\topmargin=-.5in
\textheight=9in
%\parindent=0in
%\pagestyle{plain}
%\linespread{1.3}
%% My definition
%\newcommand{\mvec}[1]{\mbox{\bfseries\itshape #1}}

%hyperref
\hypersetup{
	colorlinks=true,
	linkcolor=blue,
	filecolor=magenta,      
	urlcolor=cyan,
}

% Line spacing -----------------------------------------------------------
\newlength{\defbaselineskip}
\setlength{\defbaselineskip}{\baselineskip}
\newcommand{\setlinespacing}[1]%
{\setlength{\baselineskip}{#1 \defbaselineskip}}
\newcommand{\doublespacing}{\setlength{\baselineskip}%
	{2.0 \defbaselineskip}}
\setlength{\parskip}{1em}
\renewcommand{\baselinestretch}{1.5}
% Maths --------------------------------------------------------------------
\newtheorem{lemma}{Lemma}
\newtheorem{thm}{Theorem}
\newtheorem{definition}{Definition}
\newtheorem{example}{Example}
\newtheorem{corollary}{Corollary}
\newtheorem{remark}{Remark}
%\thispagestyle{empty}
%--------------------------------------------------------------------------

\begin{document}
	% Title page ---------------------------------------------------------------
	\thispagestyle{empty}
	\begin{titlepage}
		\begin{figure}[h]
			\centerline{\includegraphics[width=1.2in,height=1.6in]{iiitm}}
		\end{figure}
		\vspace{0.1in}
		\begin{center}
			{\Large \bf ABV - Indian Institute Of Information Technology And Management, Gwalior\\}
		\end{center}
		\begin{center}
			%{\Large \textbf{GameLib}}\\[1.2cm]
			\vspace{1.0in}
			\textbf{\huge  Project Report} \\
		\end{center}
		\begin{center}
			%\vspace{0.2in}
			%{\large Dissertation} \\
			%{\Large \it Report} \\
			%\vspace{0.3in}
			%{\large \it in\\}
			%\vspace{0.3in}
			%{\Large \bf CSE (Specialization)} \\
			%{\large \it by\\}
			{\Large \bf Topic - GameLib}\\
			\vspace{0.2in}
			{\Large \bf Object Oriented Programming} \\
			November 2020\\
			\vspace{0.5in}
			{\Large Mentor :}\\
			{\Large \bf Dr. Vinal Patel}
			\vspace{0.5in}
			\begin{multicols}{2}
				[
				{\Large Contributors :}
				]
				{\Large \bf Prateek Singh}\hspace{1in}
				{\bf (2019BCS-043)}\hspace{1in}
				{\Large \bf Divyansh Falodiya}\hspace{1in}
				{\bf (2019BCS-020)}
			\end{multicols}
			\vspace{0.4in}
			%{\large \it under the supervision of\\}
			\vspace{0.3in}
			%{\Large \bf Dr. VINAL PATEL}\\
			\end {center}
			\vspace{0.2in}
			
			%\vspace{0.2in}
			%{\large \bf Aug-Sep, 2017}
		\end{titlepage}
		% Synopsis main text -----------------------------------------------------------------
		\iffalse
		\newpage
		\setcounter{page}{1}
		\setlinespacing{1}
		\begin{center}
			{\large \bf Contents}
		\end{center}
		I hereby certify that I have properly checked and verified all the items as prescribed in the check-list and ensure that my thesis/report is in proper format as specified in the guideline for thesis preparation. \\
		
		\noindent I also declare that the work containing in this report is my own work. I, understand that plagiarism is defined as any one or combination of the following:
		\begin{enumerate}
			\item To steal and pass off (the ideas or words of another) as one's own
			\item To use (another's production) without crediting the source
			\item To commit literary theft
			\item To present as new and original an idea or product derived from an existing source.
		\end{enumerate}
		I understand that plagiarism involves an intentional act by the plagiarist of using someone else`s work/ideas completely/partially and claiming authorship/originality of the work/ideas. Verbatim copy as well as close resemblance to some else`s work constitute plagiarism.\\
		
		\noindent I have given due credit to the original authors/sources for all the words, ideas, diagrams, graphics, computer programs, experiments, results, websites, that are not my original contribution. I have used quotation marks to identify verbatim sentences and given credit to the original authors/sources.\\
		
		\noindent I affirm that no portion of my work is plagiarized, and the experiments and results reported in the report/dissertation/thesis are not manipulated. In the event of a complaint of plagiarism and the manipulation of the experiments and results, I shall be fully responsible and answerable. My faculty supervisor(s) will not be responsible for the same.\vspace{1cm}\\
		
		\noindent Signature:\vspace{1cm}\\
		Name: ANURAG SINGH \\
		Roll No.: 2018DC-02  \\
		Date: 04/03/2020   \\
		\clearpage
		%%%%%%%%%%%%%%%%%%%%%%%%%%%%%%%%%%%%%%%%%%%%%%%%%%%%%%%%%%%%%%%%%%%%%%%%%%%%%%%%%
		% Ack
		%\begin{center}
		%{\large \bf ACKNOWLEDGEMENT}
		%\end{center}
		%I am highly indebted to {\bf Supervisor name}, and obliged for giving me the autonomy of
		%functioning and experimenting with ideas. I would like to take this opportunity to express my profound gratitude to him
		%not only for his academic guidance but also for his personal interest in my report and constant support coupled with
		%confidence boosting and motivating sessions which proved very fruitful and were instrumental in infusing self-assurance
		%and trust within me. The nurturing and blossoming of the present work is mainly due to his valuable guidance,
		%suggestions, astute judgment, constructive criticism and an eye for perfection. My mentor always answered myriad of my
		%doubts with smiling graciousness and prodigious patience, never letting me feel that I am novices by always lending an
		%ear to my views, appreciating and improving them and by giving me a free hand in my report. It's only
		%because of his overwhelming interest and helpful attitude, the present work has attained the stage it has. \\
		%\\
		%Finally, I am grateful to our Institution and colleagues whose constant encouragement served to renew my spirit,
		%refocus my attention and energy and helped me in carrying out this work.\\ \\ \\ \\
		%\vspace{0.7in}
		%\textbf{Date:}
		%\hspace{3.8in}
		%\textbf{Candidate name}
		%\clearpage
		%%%%%%%%%%%%%%%%%%%%%%%%%%%%%%%%%%%%%%%%%%%%%%%%%%%%%%%%%%%%%%%%%%%%%%%%%%%%%%%%%%
		
		
		\topmargin=-1in    
		\textheight=24cm  
		\oddsidemargin=0pt % leftmargin is 1 inch
		\textwidth=6.5in   % textwidth of 6.5in leaves 1 inch for right margin
		
		\begin{center}
			{\large \bf ABSTRACT}
		\end{center}  
		
		Recently, wireless sensor networks (WSNs) are gaining interest among the research community due to their vast applications in the field of health care, defense, household environment,etc. Fast convergence rate, low complexity, less estimation error as well as high energy efficiency are the primary design features of communication protocols and physical layer technologies used in WSNs. Because of limited power supply, computational power and memory capacity of sensor nodes, power management techniques and less power consuming relay policies should be used in order to achieve acceptable signal to noise ratio as well as better quality of service. Most of these parameters are affected by the presence of modelling error in the channel estimation process. Therefore the performance of WSNs can be improved by reducing the channel estimation error. Therefore, in this thesis, a set of
		novel channel estimation algorithms that can achieve both faster convergence and smaller steady-state error will be designed.  \\
		\\
		\\
		%{\it \textbf{ Keywords: Channel estimation, fast convergence, high energy efficiency, convex combination, set-membership, error bound.}}\\
		
		
		\clearpage
		\fi
		%\listoftables
		
		\tableofcontents
		\clearpage
		%%%%%%%%%%%%%%%%%%%%%%%%%%%%%%%%%%%%%%%%%%%%%%%%%%%%%%%%%%%%%%%%%%%%%%%%%%%%%%%%%%
		\section{Introduction}
		\setlength{\parindent}{10ex}
		Games are an essential part of every person's life. Be it a kid, a youngster or an elder, everyone is aware of and have played various games. Games have always been a part of our culture and they still are. Games like 'Chaupar', 'Ashtapad', 'Krida-Patram' and many more have always been referenced in many historical texts. Be it any era, games have always been present to entertain people and have kept on evolving to suit the human mind.
		\subsection{GameLib}
		\begin{flushleft}
			\setlength{\parindent}{10ex}
			GameLib is a software application which provides the user with plenty of games to play. It is written completely in the C++ programming language and is built specifically for the Windows(x64) Operating System\footnote{This is subject to change.}. \par
			The application utilizes a third party library known as Simple DirectMedia Layer which provides us with low level access to audio, keyboard, mouse, and graphics hardware via OpenGL and Direct3D. \par
			\noindent There are, at present, five games available in GameLib :
			\begin{itemize}
				\item {\bf Tic-Tac-Toe}
				\item {\bf Pong}
				\item {\bf Hangman}
				\item {\bf Snake}
				\item {\bf Flappy Bird}
			\end{itemize}
		\end{flushleft}
		
		\clearpage
		
		\subsection{Features}
		\begin{flushleft}
			\setlength{\parindent}{10ex}
			The features of this project are something that differentiates it from other similar applications. Some remarkable features include :
			\begin{itemize}
				\item {\bf Authentication} - The application keeps track of the user who has registered to it and lets the user create a password for it during registration. Only a authenticated user can play the games that are included in the application.
				\item {\bf Security} - The application is completely secure and the passwords are encrypted so that no other user can alter your private data.
				\item {\bf Multiple Users} - The application has the ability to create various user profiles and manage them seperately, thus allowing multiple users to access the application.
				\item {\bf AI} - Certain two player games also have the feature of AI to play the chance of a player so that even a single player can be entertained by the games. This feature is remarkable because it makes sure that the computer never misses a chance to have an equal or upper-hand on the user.
				\item {\bf User Interface} - The user interface for the application is completely graphical and is elegant enough to keep the user engaged. The visuals of the application are gripping enough and they come along with a dark theme to provide the best user experience.
			\end{itemize}
		\end{flushleft}
		
		\clearpage
		
		\section{Principles and Concepts}
		\subsection{Game Programming}
		\begin{flushleft}
			\setlength{\parindent}{10ex}
			 This aspect of programming deals with the core features of a game. Rendering, game physics, collisions, frame rates, etc, are all of a part of this aspect. As far as GameLib is concerned, it makes a thorough use of the very basics of game programming. \par
			 For example, games like {\it Flappy Bird, Snake}, etc, have the feature of collision detection to detect any collision between the walls(say) and the player and updates the scores accordingly.
		\end{flushleft}
		\subsection{Object Oriented Programming}
		\begin{flushleft}
			\setlength{\parindent}{10ex}
			Object Oriented Programming is the foundation layer over which the whole of GameLib stands. It is the most important aspect of the application. Programming features like modularity, object instances, encapsulation, etc, are only possible because of OOPS. The application make use of most of these features provided by OOPS. It is because of OOPS's features like modularity and abstraction that the source code for the application is easily understandable and can be upgraded to better suit the implementation without much hassle. \par 
			The most important feature of OOPS (in terms of use) is {\bf Modularity}. It is this feature that made sure the code was neat and understandable to even a non-programmer. It made sure that the source code can be divided into a bunch of different modules with each having its own purpose and thus decreasing the hassle required for reusing particular functions and methods. \par 
		\end{flushleft}
	
		\clearpage
		
		\section{Significance of the project}
		\begin{flushleft}
			\setlength{\parindent}{10ex}
			GameLib is a project that we started a month ago to understand and apply our knowledge of Object Oriented Programming and its various domains. Little did we know that we would be able to increase our scope of knowledge to such a great extent. Working on GameLib has been a remarkable journey for us not only because we were able able to implement our skills and knowledge but also because we were able to garner and grow our existing knowledge of such a vast topic. \par
			Building GameLib from absolute scratch has been a really great experience for us because not only did it make us understand more about Object Oriented Programming but also introduced us to various aspects of programming in general. This project has been a real eye-opener for us. \par 
			{\it While this may be the first version of GameLib and may not be the best one out there, we will make sure that we keep upgrading it with our growing knowledge of the domain and will try to implement some really important features which may have been missed in this one. You can build the source code from our github repository \href{https://github.com/DivyanshFalodiya/GameLib}{here}}
		\end{flushleft}
		\clearpage
		
		%%%%%%%%%%%%%%%%%%%%%%%%%%%%%%%%%%%%%%%%%
		\clearpage
		
	\end{document}
	
